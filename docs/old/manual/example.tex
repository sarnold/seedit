%#!platex manual.tex
\section{Example}
\subsection{Adding policy to existing policy}
You can generate simplified policy like audit2allow by audit2spdl utility.
Typical usage is following.\\
\begin{verbatim}
$ su - (select sysadm_r)
# audit2spdl -a -l -v
\end{verbatim}
Then, policy is outputted to console. 
Make sure whether everything is correct.
Then, when you want to add policy to file,
\begin{verbatim}
# cd  /etc/selinux/seedit/src/policy/
# audit2spdl -a -l -o ./simplified_policy
# make diffrelabel
\end{verbatim}
When you are not running auditd, use -d option instead of -a option.
However, file path for {\it allow} rule is sometimes incorrect in -d
option. So it is recommended to enable auditd. \\
Notice: audit2spdl is still in progress. For some log, audit2spdl can
not output configuration. Please tell me if you find problem.

\subsection{Configuring vsftpd}
As an example of configuration of daemon for that policy is not
prepared,  let's configure vsftpd using
simplified policy. In this example, configure Anonymous ftp access.\\
By default, the domain of vsftpd is {\it initrc\_t }.
{\it initrc\_t } is a domain for scripts under /etc/rc.d . vsftpd
is executed by /etc/rc.d/init.d/vsftpd(the domain is initrc\_t) and
inherits the domain.\\ 
However, this is not secure. Because initrc\_t has a lots of access
rights(see /etc/selinux/seedit/src/policy/simplified\_policy/initrc\_t.a
).\\
In following current directory is {\it
/etc/selinux/seedit/src/policy}, and permissive mode\\
\begin{verbatim}
login: root
....
# newrole -r sysadm_r
# id -Z
root:sysadm_r:sysadm_t
# cd  /etc/selinux/seedit/src/policy
# setenforce 0
\end{verbatim}
And for detail of syntax, see \ref{sec:spec} .

\subsubsection{Create domain for vsftpd}
Let's give vftpd {\it vsftpd\_t} domain.
\begin{enumerate}
 \item Create configuration file\\
Create {\it simplified\_policy/vsftpd\_t.a }. 
 \item Configure domain transition\\
 In {\it simplified\_policy/vsftpd\_t.a} write following.
\begin{verbatim}
# simplified_policy/vsftpd_t.a
{
domain vsftpd_t;
domain_trans initrc_t /usr/sbin/vsftpd;
}
\end{verbatim}

In line 2, you've defined domain vsftpd\_t.  In line 3, you've configured
       domain transition, parent domain is initrc\_t, entry point is /usr/sbin/vsftpd.

\end{enumerate}

\subsubsection{Test domain transition}
When you edit configuration, you must use make command to indicate
kernel change of configuration(See \ref{sec:loadpolicy}). In this case, type as below.\\
\begin{verbatim}
# make diffrelabel
\end{verbatim}
Usually {\it make diffrelabel } is enough.\\
Restart vsftpd and check the domain of vsftpd.
\begin{verbatim}
# /etc/init.d/vsftpd restart
# ps -eZ
...
root:system_r:vsftpd_t          13621 pts/1    00:00:00 vsftpd
...
\end{verbatim}
You can see that the domain of vsftpd is {\it vsftpd\_t}. Domain transition
is successful.

\subsubsection{Protect files related to vsftpd}
 Protect files related to vsftpd\\
       If you want to protect files related to domain, the best way is {\it
       deny } in global. In this case, let's protect /etc/vsftpd and
       /var/ftp.  Add following in simplified\_policy global. Note you
       have to add between \{ and \}.
       \begin{verbatim}
# In simplifed_policy/global
deny /etc/vsftpd;
deny /var/ftp;
       \end{verbatim}
       And 
\begin{verbatim}
# make diffrelabel	
\end{verbatim}
       As a result, if some domain want to access /etc/vsftpd and
       /var/ftp, it must be allowed explicitly. e.g: If httpd\_t want to read
       /etc/vsftpd, {\it allow /etc/vsftpd r;} must be described in
       httpd\_t, if {\it allow /etc r;} is described, access to
       /etc/vsftpd is not allowed. deny is useful to mark important
       files.

\subsubsection{Give access rights to vsftpd\_t}
The default access right for vsftpd\_t is inherited from
simplified\_policy/global. It is not enough, you have to add
configuration.  The best way to know what
       is necessary is to test vsftpd on permissive mode and see SELinux
       log. Then use {\it audit2spdl -a -l} command(Detail is skipped in
       this document)
Below is a policy for vsftpd\_t. 
\begin{verbatim}
# simplifed_policy/vsftpd_t.a
     1  {
     2  domain vsftpd_t;
     3  domain_trans initrc_t /usr/sbin/vsftpd;
     4  # access to files related to vsftpd
     5  allow /etc/vsftpd r,s;
     6  allow /var/ftp r,s;
     7  allowonly /var/log r,w,s;
     8  # allow to communicate with syslog
     9  allow dev_log_t r,w,s;
    10  allowcom -unix syslogd_t;
    11  # allow to use tcp 20 and 21
    12  allownet;
    13  allownet -connect;
    14  allownet -tcp -port 20;
    15  allownet -tcp -port 21;
    16  #
    17  allowadm chroot;
    18  }
\end{verbatim}
After writing this, 
\begin{verbatim}
# make diffrelabel
\end{verbatim}
Let's review the file.
\begin{itemize}
 \item Line 5 to 7\\
       These are configuration to access files related to vsftpd. In line 5 and
       6, giving access rights to read vsftpd configuration files and
       ftp public directory. \\
       Pay attention to line 7.
       \begin{verbatim}
allowonly /var/log r,w,s;
       \end{verbatim}
       In this, we want to allow to write /var/log/xferlog. If we could
       configure,
\begin{verbatim}
allow /var/log/xferlog r,w,s;
\end{verbatim}
       this would be the best. However, /var/log/xferlog may be
       deleted by administrater, and when re-created the SELinux label
       information is lost. So we can not control access to
       /var/log/xferlog. 
       So we used {\it allowonly /var/log r,w,s}. In this vsftpd\_t can write
       all files on /var/log/, but can not write files on child
       directories. This is better than {\it allow /var/log r,w,s;}(This
       allows write access to all files under /var/log including child directories).
       Similally, for /tmp, /var/run, you can not controll access
       per-file, in those directories, files are deleted and re-created,
       SElinux label information may be lost.\\
       If you have a enough knowledge of SELinux, you can use {\it allow
       file exclusive label;} This configures SELinux's file type
       transition. You can configure access control to files that are
       deleted and re-created. For detail see \ref{section:allow}.\\

 \item Line 9,10\\
       Those are to communicate syslogd. If you want to communicate with
       syslogd, always include these two lines.
       
 \item Line 12-15\\
       These are to communicate via tcp 20 and 21.
\end{itemize}

\subsubsection{Give access rigths to initrc\_t}
initrc\_t is a type for start up script(/etc/init.d/vsftpd). 
This requires read access to /etc/vsftpd. But access to this file is
denied in global, so you have to allow explicitly.
\begin{verbatim}
#add to simplified_policy/initrc_t.a
allow /etc/vsftpd r,s;
\end{verbatim}
Then, 
\begin{verbatim}
# make diffrelabel
\end{verbatim}

\subsubsection{Test again}
Test in permissive mode and see access log. 
If no deny is outputted, then test in enforcing mode.


