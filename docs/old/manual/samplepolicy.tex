%#!platex manual.tex
\section{Sample Simplified Policy}
This section describes information about sample simplified
policy. If you are in a hurry, read \ref{sec:default}, \ref{sec:middlepolicy} and \ref{sec:loadpolicy}.
\subsection{Default configuration in simplified policy}\label{sec:default}
\begin{itemize}
 \item Supported services in version 1.3.3\\
       Supported services in sample policy are acpid, auditd, automount,
       cannaserver, crond, dbus, dhclient, dovecot, gdm,  hald,
       httpd, iiimd, klogd,  mysql, named, newrole, pcmcia, postfix,
       sendmail, smbd, sshd, syslogd, cron ,udev, xfs, xinetd and X
       window system. 
 They run as {\it servicename}\_t domain.
 \item RBAC configuration\\
      Three roles sysadm\_r, staff\_r, user\_r are prepared by
       default.\\
       \begin{itemize}
	\item sysadm\_r\\
	      Can do everything. only user root can use this.
	\item staff\_r\\
	      Have limited access rights. Can use su command. can read
	      /root. only root can use by default.
	\item user\_r\\
	      Have limited access rights. Can not use su
	      command. Default role for every user.
       \end{itemize}
\end{itemize}

\subsection{Contents of directory for simplified policy}
 Sample simplified policy is located at
 "/etc/selinux/seedit/src/policy/". 
Some directories and files are located there.
\subsubsection{simplified\_policy}\label{sec:middlepolicy}
  This is the most important. At the directory sample simplified policy is stored. For detail of syntax for simplified policy, see \ref{sec:spec}.\\
  Sample simplified policy are described in {\it global}, and {\it domain}.te 
  \begin{itemize}
  \item global\\
    This is a configuration commonly used by all domains. Be careful
	that some access rights(For example, tmpfs usage, tty device
	access) are granted by default to show that SELinux can become
	easy.
  \item {\it domain}.a\\
    Here configuration for domains are described. For example,in file
	{\it httpd\_t.a}, configuration for {\it httpd\_t} domain is
	described.
   \item all\\
	 global, and *.a are jointed. converter reads this file. This
	file is automatically generated. Do not edit.
  \end{itemize}
 
\subsubsection{Makefile}
 This is Makefile to compile simplified policy and to load policy to
 kernel. See \ref{sec:loadpolicy}.
\subsubsection{base\_policy}\label{sec:basepolicy}
Files in this directory is used by converter to generate SELinux
policy. Usually you do not have to touch them.
  \begin{itemize}
  \item default.te \\ 
    This file is useful. Statements described in this file is included
	in policy to be generated by converter. 
 In addition, you can write original SELinux's rules here. To write auditallow rule is a good idea. But allow rules must not be written here, because it can break the security of generated policy.

   \item unsupported.te\\
Statements described in this file is included
	in policy to be generated by converter. In this file,
	unsupported permissions are described. Permissions described in
	this file is allowed to all domain. Do not edit them.
   \item attribute.te \\  
	In this file, attributes used in policy to be generated by converter is described. Do not edit.
   \item types.te\\
	In this file, types used in policy generated by converter is described. Do not edit.\\
  The meaning of below files are the same as original SELinux's
	policy. Do not edit them.
	

	\begin{itemize}
	 \item genfs\_contexts\\
	       File system that are unlabeled are not supported.
	 \item security\_classes
	 \item access\_vectors  
	 \item initial\_sid\_contexts
	 \item fs\_use 
	 \item initial\_sids
	\end{itemize}
  \end{itemize}

\subsubsection{macros}
Macros to generate SELinux policy is stored. Converter generates policy
including macros. In make, macros are processed by m4 and policy.conf is
generated. 

\subsubsection{sepolicy}
 Generated SELinux policy is written in this directory. 
\begin{itemize}
 \item test.conf\\
 Policy that includes macros.
 \item policy.conf
 Policy file that is processed by m4. This is understandable by checkpolicy.
 \item file\_contexts\\
 This will be installed in /etc/selinux/seedit/contexts/files/file\_contexts.
\end{itemize}

\subsection{To load policy}\label{sec:loadpolicy}
 If you modify policy in simplified\_policy, you must load the modified
 policy to kernel. There are some targets in Makefile. \\
First, you have to go /etc/selinux/seedit/src/policy.
\begin{itemize}

\item make reload\\
 This converts simplified policy in simplified\_policy dir into SELinux policy.\\
(1) policy.conf and file\_contexts are generated in sepolicy dir. \\
(2) binary policy is created by checkpolicy from generated policy.conf.\\
(3) binary policy is installed in /etc/seedit/policy, file\_contexts is installed in /etc/seedit/contexts/policy. And contents of base\_policy/contexts is installed in /etc/seedit/contexts.\\
(4)  binary policy is loaded by load\_policy
\item make relabel\\
 After {\it make reload}, {\it fixfiles restore} runs.

\item make diffrelabel\\
 This is very useful. {\it make relabel} relabels all files, and it
      takes a lot of time. On the other hand, {\it make diffrelabel}
      relabels only files whose label have changed. So it takes only
      some seconds. When you load policy, {\it make diffrelabel} is recommended.
\end{itemize}

\section{Option of converter}
converter is usually run by make, you do not have to use converter directly.
Following is only for reference.\\
Usage : converter -i {\it infile} -b {\it base policy dir} -o
{\it policyfile} -f {\it file\_context}\\
\begin{itemize}
 \item {\it infile}\\
       Simplified policy file. In make this is all.
 \item {\it base policy dir}
       Directory for base\_policy.
 \item {\it policyfile}
       Generated policy file name.
 \item {\it file\_contexts}
       Generated file\_contexts fine name.
\end{itemize}
