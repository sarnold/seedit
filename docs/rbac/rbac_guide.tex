\documentclass{article}
\title{SELinux Policy Editor RBAC(Role Based Access Control) guide (for Ver 2.1))}
\author{Yuichi Nakamura \thanks{himainu-ynakam@miomio.jp}}
\begin{document}
\def\labelenumi{(\theenumi)}
\maketitle
\tableofcontents
\newpage

This document describes how to use RBAC(Role Based Access Control) in seedit.

\section{What is RBAC}
\subsection{Overview}
SPDL supports configuration of RBAC(Role Based Access Control).
In default policy, the domain for login user is unconfined\_t.
So, the behavior of user is not confined by SELinux.\\
To increase security of login user, RBAC is useful.
By using RBAC, you can restrict behavior of users by assigning role to
user. For example, you can assign role webmaster\_r to user webmaster,
and give him rights to do web master work. 

\subsection{How RBAC works in SELinux}
How RBAC works is composed of 2 parts, one is assign role to user, 
second is assign domain to user shell.

\begin{enumerate}
 \item Assign role \\
  When user logs in from login programs(login, sshd, gdm), login programs
  assign role to users. The rule that describes what kind of roles the
  user is allowed to use, is described in policy.
  Login programs assign roles referring to policy.For example, if user
       webmaster is allowed to use webmaster\_r role,  login program assign webmaster\_r to user webmaster.\\
 \item Assign domain\\
  Role is only strings, 
  to confine behavior of user, domain must be given to user shell. \\
  When user shell is launched domain is given according to role. For
       example, when user webmaster login as webmaster\_r role,
       webmaster\_t domain is given to user shell. webmaster\_t domain
       is configured to be allowed homepage admin works.

\end{enumerate}


\section{Enable RBAC}
To enable RBAC, use following command.
\begin{verbatim}
# seedit-rbac on
(It takes some minutes)
# reboot
\end{verbatim}
By above commands, files necessary to configure RBAC is moved from
/etc/seedit/policy/extra to /etc/seedit/policy, and seedit-load is
run.Reboot is necessary because some domains become invalid, to fix this
you have to reboot.\\

To disable RBAC, use following.
\begin{verbatim}
# seedit-rbac off
# reboot
\end{verbatim}

\section{Default RBAC Configuration}
Following 3 roles are defined by default.

\begin{itemize}
 \item sysadm\_r\\
       It is role for administrator. It can work as unconfined
       domain sysadm\_t. By default, only root can login as the role.
 \item staff\_r \\
       It is role, to do not administrative work for administrative
       user.
       By default, only root can login as the role.
 \item user\_r \\
       It is a role for normal users. By default, user\_u can login as
       root.
       user\_u is a user that is not configured RBAC, by default users
       except root.
       
\end{itemize}

\begin{itemize}
 \item Attention\\
       \begin{enumerate}
	\item  su command can not be used, except sysadm\_r\\
	\item  Only user who can use sysadm\_r can login from gdm.\\
	       You need to add a lots of configurations to login from
	       other roles, and it may decrease security. By default
	       only root user can login from X. If other users want to
	       login from X, you have to allow to use sysadm\_r, but be
	       careful, because behavior of such users are not confined
	       by SELinux.
	\item Configure to use sysadm\_r\\
	       Usually, you will not login root user directly. You
	       will use su, you have to allow some user to use sysadm\_r.
	       For example to allow user ynakam to use sysadm\_r, add following after {\it
	       user root;} in /etc/seedit/policy/sysadm\_r.sp.\\
	       \begin{verbatim}
		user ynakam;
	       \end{verbatim}

       \end{enumerate}
\end{itemize}


\section{Login by RBAC}
\subsection{Check role}
When you enable RBAC, role are given to login user. 
You can see it by id command.\\
When you login as root user,  staff\_r role is given.
root is allowed to use staff\_r and sysadm\_r, but login program give
staff\_r role.
\\
Let's see it by id command.
\begin{verbatim}
# id 
uid=0(root) gid=0(root) groups=0(root),1(bin),2(daemon),
3(sys),4(adm),6(disk),10(wheel) 
context=root:staff_r:staff_t
\end{verbatim}
\begin{verbatim}
context=root:staff_r:staff_t	
\end{verbatim}
shows role, staff\_r is role. User shell is given domain according to
role, in this case staff\_t. Inside SELinux, domain is used for access
control.\\
staff\_r is given little access rights. You can not do any
administration work in this role.\\
For example, you can not access homepage.
\begin{verbatim}
# cat /var/www/html/index.html
Permission denied
\end{verbatim}

\subsection{Change role}
To do administrative work, you have to switch role to sysadm\_r role.
You can do it by newrole command, like below.
\begin{verbatim}
# newrole -r sysadm_r
Authenticating root
Password:
\end{verbatim}
You have to enter password of current user(this case root).
Then check role by id command.
\begin{verbatim}
# id 
uid=0(root) gid=0(root) groups=0(root),1(bin),2(daemon),
3(sys),4(adm),6(disk),10(wheel) 
context=root:sysadm_r:sysadm_t
\end{verbatim}
Role is  sysadm\_r. Domain of user shell is sysadm\_t. sysadm\_t is
unconfined domain, so you can do any work.\\
To switch role, the user must be allowed to use the role, if the user is
not allowed to use the role, newrole will fail.\\
To allow user to use role, see next section.


\section{Configuration elements for RBAC}
To configure RBAC, it is better to understand how to describe.
\subsection{role and user statements}
To configure RBAC, there are 2 SPDL configuration elements, {\it role} and
{\it user}.
\begin{itemize}
 \item role {\it name  of role}\\
       This means configurations is going to be done for specified
       role. Access right is given to corresponding domain. And the
       configuration filename must be {\it role name}.sp.
 \item user {\it user name}\\
       This allows user to use role.  user\_u user name is special, it is
       the default role of users that is not configured to use
       domain.How to allow to use sysadm\_r, see section \ref{sec:allowsysadm}.
\end{itemize}

\subsubsection{Example}
Let's see example.
\begin{verbatim}
1:{
2:role webmaster_r;
3:user webmaster;
4:allow /var/www/** r,w,s;
}	
\end{verbatim}

Line 2 means, configurations between \{\} is for webmaster\_r role.\\
Line 3 means, user name webmaster can use webmaster\_r role.\\
Following, access rights are given to domain webmaster\_t
domain(Remember that webmaster\_r role behaves as webmaster\_t domain in SELinux system).
Line 4 means, webmaster\_t domain(This equals user that logined as
webmaster\_r role)is allowed to read, write under /var/www.\\
\subsubsection{user\_u user name}
Let's see example  of user\_u user name.\\

\begin{verbatim}
Assume all configurations for RBAC are following.
* In sysadm_r.sp
{
role sysadm_r;
user root;
..
}	
* In webmaster_r.sp
{
role webmaster_r;
user webmaster
..
}
* In user_r.sp
{
role user_r;
user user_u;
..
}
\end{verbatim}
In above, 3 roles are configured. 
You can see, user root and webmaster are assigned role.
In this case user\_u is {\it all users except root and webmaster}.


\subsection{Home directories}

To configure access control to home directories, we could use
\textasciitilde / . In normal domain, it means all users home
directories. But, for configuration of role, the meaning is different.\\
The rule is simple:
\begin{verbatim}
~/ means home directories for users that can use role	
\end{verbatim}
Let's see example.
\begin{verbatim}
{
1:role webmaster_r;
2:user web1;
3:user web2;
4:allow ~/** r,w,s;
\end{verbatim}
In this case, line 4 is allow to write home directories for user1 and
user2.
So, when user web1/web2 login as webmaster\_r role, they can read write
their home directories, but can not access other users home directories.

\begin{verbatim}
{
1:role user_r;
2:user user_u;
3:allow ~/** r,w,s;
\end{verbatim}
user\_u is supported, too. Line 3 means home directories for user\_u users.

\subsection{Allow user to use sysadm\_r role}\label{sec:allowsysadm}
You can easily allow to use sysadm\_r.
Open /etc/seedit/policy/sysadm\_r.sp(Configuration file for sysadm\_r
role).
add following, after {\it user root;}
\begin{verbatim}
user <username you want to allow to use sysadm_r>;	
\end{verbatim}
and seedit-load.


\section{Creating new role}
The best way to understand RBAC is to create  new role. Let's see it by
example.
Here, we will create new role for webmaster, the name is webmaster\_r.
And assign the role to user whose name is webmaster.
\subsection{Create uid=0 user}
The uid for  user that does some administration work, must be 0.
It is to pass Linux permission check.\\
Following commands create user webmaster as uid=0.
\begin{verbatim}
# useradd -u 0 -o webmaster
# passwd webmaster
\end{verbatim}

\subsection{Create template}
You can create template configuration by seedit-template command.\\
The usage is following.
\begin{verbatim}
seedit-template -r <role> -u <user> -o <output directory>
\end{verbatim}
If you specify -o option, configuration is written to file, before
writing to file, run command without -o option to make sure.\\


Following is example of generating configuration for webmaster\_r role.

\begin{verbatim}
# seedit-template -r webmaster_r -u webmaster
{
role webmaster_r;
user webmaster;
include user_common.sp;
include common-relaxed.sp;
allow ~/** r,w,s;
allowpriv part_relabel;
allowpriv dac_override;
allowpriv dac_read_search;
}
\end{verbatim}
Template configuration is generated. user webmaster can use webmaster\_r
role. By include common configurations to behave as login user is
imported, system critical access rights are not allowed here.
And webmaster\_t is allowed to access user webmaster's home
directory(When user webmaster login as webmaster\_r he can access his
home directory).\\
3 allowpriv are outputted, this is usually needed to do administration work.
\begin{verbatim}
allowpriv part_relabel;	
\end{verbatim}
This is necessary to use restorecon. You can use restorecon to files those
webmaster\_r is writable. If you do not use restorecon, delete this.
\begin{verbatim}
allowpriv dac_override;
allowpriv dac_read_search;
\end{verbatim}
Those are necessary to skip Linux permission check.

\subsection{Write to file}
You can write above configuration by -o option of seedit-template.
\begin{verbatim}
# seedit-template -r webmaster_r -u webmaster -o /etc/seedit/policy/	
\end{verbatim}
Configuration is written to /etc/seedit/policy/webmaster\_r.sp


\subsection{Add permissions for administration }
Then, add permissions to administrate web page.
You need write permission under /var/www.
So, add 
\begin{verbatim}
allow /var/www/** r,w,s;
\end{verbatim}
before \}.

\subsection{Load and test}
Let's load configuration by seedit-load.
\begin{verbatim}
#seedit-load
\end{verbatim}
And test to login.
Login as webmaster.
Let's see role.
\begin{verbatim}
#id
uid=0(root) gid=502(webmaster) groups=502(webmaster) 
context=webmaster:webmaster_r:webmaster_t
\end{verbatim}
Now you are webmaster!\\
You can upload file to /var/www/html from your home directory.
If you find something is denied and it is necessary, test in permissive
mode, and  add
configuration to /etc/seedit/policy/webmaster\_r.sp using audit2spdl
like normal domain.

%�桼�����Ǻ�ä��顢restorecon��ɬ�ס���


\end{document}