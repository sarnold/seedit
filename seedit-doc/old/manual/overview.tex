%#!platex manual.tex
\section{About this document}
 This document is a reference manual for simplified policy(version 1.4.0). About install, see How
to install.  Note that this document is under construction, the contents
will change.

\section{Overview}
  SELinux\cite{selinux} implements fine-grained Mandatory Access Control
  on Linux. However, the access control is too fine-grained and its
  policy tends to be too complicated. So it is very difficult to understand and configure policy.
  "Simplified policy" is a policy written in Simplified Policy Description
  Language(SPDL). SPDL reduces the number of policy
  description by hiding type label from users and integrating object
  classes and access vectors. User can use SELinux system easily by
  using this. For example, if httpd\_t domain want to read /var/www and
  use tcp port 80, the configuration is like below.\\
\{\\
domain httpd\_t;\\
allow /var/www r,s;\\
allownet -tcp -port 80;\\
\\
  Simplified policy was originally developed as a part of SELinux Policy
  Editor\cite{selpepaper}\cite{selpetalk} by Hitachi
  Software\cite{hitachisoft}. Now it is maintained in SELinux Policy
  Editor Project\cite{selpehomepage}. Fedora Core4 and 3 are supported.  Simplified Policy does not affect existing SELinux, you can go back default SELinux easily. Feel free to try! 

\section{Components of Simplified policy}
Simplified Policy is composed of two {\it must} components,
Converter and sample policy. In addition, there is an optional component, GUI(SELinux Policy Editor).

\begin{enumerate}
\item converter \\
 A compiler for SPDL. This reads simplified
      policy(written in SPDL) and generates SELinux policy understandable by m4 and checkpolicy.
\item sample policy\\
 A sample simplified policy. 

\item GUI(SELinux Policy Editor)(optional, not exist for 1.4.0)\\
 A Web-based GUI to edit simplified policy. 
        By using GUI, SELinux becomes much easier. Simplified policy with
      GUI is called {\it SELinux Policy Editor}.
\end{enumerate}

\section{Notice}
 \begin{itemize}
  \item Syntax will change in next version. We are reviewing the
	security  of simplified policy. As a result, syntax may change.
 \end{itemize}

 
