\documentclass{article}
\title{Specification of Simplified Policy Description Language(SPDL) ver 2.0}
\author{Yuichi Nakamura \thanks{The George Washington University, Hitachi Software, ynakam@gwu.edu}}
\begin{document}
\maketitle
\tableofcontents
\newpage
This document describes syntax and meaning of SPDL configuration elements.



\section{Overview}
\subsection{Feature}
The feature of SPDL is hiding labels, and reducing number of
permissions.
\begin{itemize}
 \item Hiding labels\\
       SPDL does not use types to configure access control. You can use
       file name and port number to configure.
 \item Reduce number of permissions\\
       There are too many permissions in SELinux, so SPDL reduces number
       of permissions by removing permissions and integrating
       permissions.
       Permissions that does not have security impact is
       removed. Permission removal is implemented by allowing that
       permission to all domains. Integrating permission means, treating
       set of permissions as one permission. For example r permission
       for file integrates SELinux perissions related to file read.
       For detail of what kind of permissions are removed, integrated  see {\it
       Integrated/unsupported permissions in Simplified Policy } in
       http://seedit.sourceforge.net/doc/permission\_integrate/.
\end{itemize}


\subsection{Overview of SPDL congiguration elements}
Configuration elements are following.
\begin{itemize}
 \item Giving domain to applications\\
       To give domain, SPDL  supports to configure domain transition by
       {\it domain\_trans}. For easier configuration, SPDL has {\it
       program} element.
 \item RBAC\\
       SPDL supports RBAC. {\it role} and {\it user} elements do that.
 \item Access control to file\\       
       {\it allow/deny} are SPDL elements that enables to configure
       access control to normal files. {\it allowdev} exists for device
       files, and {\it allowtmp} exists for temporaly(dynamically
       created) files.
 \item Access control to network\\
       {\it allownet} does this.
 \item Access control to IPC\\
       {\it allowcom} does access control to IPC and signal.
 \item Access control to other privilege\\
       Other important OS operations that is not restricted above can be
       configured by {\it allowpriv}.
\end{itemize}

\subsection{Default deny rule}
Domain and roles are denied all permissions unless allowed by SPDL.

\subsection{Terms}
\begin{itemize}
\item Domain \\
 Domain is the same as domain in SELinux. It is attached to process by domain transition.
\item Role\\
 Role in simplified policy language is simplified. Role is identified
      with a domain for user shell. In simplified language, we describe access
      rights for role. In fact, it is giving access rights for user
      shell of the role. For example, when you give access right for
      {\it sysadm\_r}, access right is given to 
{\it sysadm\_t}(Domain for user shell of {\it sysadm\_r}).\\
 Note that in generated SELinux policy, all roles can type every types.
      There is no syntax corresponding to {\it role:x:types:y} in simplified language.
\end{itemize}                  

\section{Structure of configuration by simplified language}
Configuration is composed of sections. In each section, access control for domains/roles are described. Section begins with $\{$ and ends with $\}$. 

\subsection{Syntax of section}

$\{$  (begin of section)\\
$domain/role$  (declare domain or role, one domain or role can be declared in one section) \\
$users$  (This can be used only for role)\\
$domain\_trans$|$program$ (Configure domain transition)\\
$allow/deny$  (Describe access control for files)\\
$allowxxx$   (Describe access control for resources other than file)\\
$\}$\\

\section{Including configuration from other file:include}
you can include configuration from other file by {\it include}
statement.
Syntax is following.\\
include {\it filename};
The include path can be specified by -I option of seedit-converter, by
default it is /etc/selinux/seedit/src/policy/simplified\_policy/include.


\section{Declaring domain and role}
\subsection{declare domain:domain}
\begin{enumerate}
 \item Syntax \\
      domain {\it domainname} ;
 \item Meaning\\
       Declare domain. All configuration in a section is done for
       the domain declared by this statement. 
 \item Constraints\\
Domain name must end with {\it \_t}. This statement can not be used twice in one section. 
\end{enumerate}

\subsection{declare role:role}
\begin{enumerate}
 \item Syntax\\
    role {\it  rolename } ;
 \item Meaning\\
     Declare role.  {\it  rolename } is associated to user by using
       {\it user} statement as shown below.
 \item Constraints\\
       {\it rolename} must end with {\it \_r}.
\end{enumerate}

\section{Configuring RBAC:user}
\begin{enumerate}
 \item Syntax\\
       user {\it user name};
 \item Meaning\\
       Define users who can use the role. 
 \item Example\\
       \{\\
       role user\_r;\\
       user root;\\
       user ynakam;\\
       ....\\
Above means user root and ynakam can use user\_r.
\item Constraints\\
       It can be used only section where role is declared. 

\end{enumerate}

\section{Configuring domain transition:domain\_trans}
\subsection{Domain transition:domain\_trans}
\begin{enumerate}
 \item Syntax\\
    domain\_trans {\it parentdomain} {\it filename-of-entrypoint};
 \item Meaning\\
    This defines how domain is attached to process. 
 \item Example\\
   \{\\
   domain httpd\_t;\\
   domain\_trans initrc\_t /sbin/httpd;\\
   ....\\
   Above means that when process(domain: initrc\_t) executes
       /sbin/httpd, /sbin/httpd runs as httpd\_t domain.
 \item Note\\
       Dynamic domain transition can be configured by omitting entry
       point. For example,  
       \{\\
       domain httpd\_t;\\
       domain\_trans  initrc\_t;\\
       
       means, dynamic domain transition from initrc\_t to httpd\_t is allowed.

\end{enumerate}

\subsection{Simplified domain transition:program}\label{sec:program}
\begin{enumerate}
 \item Syntax\\
    program {\it path-to-program};
 \item Meaning\\
    {\it path-to-program} is attached domain in normal case.   By this, {\it path-to-program} is attached domain launched from
       command line, and /etc/init.d scripts.That is to
       say, allow domain transition from unconfined domain. unconfined
       domain means, the domain that is configured {\it allowpriv all;}.
    However, domain transtion from authentication program domain(such as
       domains for su,login,sshd) is not configured. Which domains are
       regarded as authentication\_domain is configured in
       authentication\_domain field in converter.conf.\\
 \item Example\\
1)
\begin{verbatim}
{
domain httpd_t;
program /usr/sbin/httpd;
}
/usr/sbin/httpd is attached httpd_t domain when launched from command
		line and /etc/init.d script.
\end{verbatim}
 \item Note\\
    This element is intended to be used in relaxed policy. This does not
       mean nothing in strict policy, because {\it
       allowpriv all;} is not used.
\end{enumerate}


\section{Configuring access control to normal files:allow/deny}
\subsection{allow}\label{section:allow}
 \begin{enumerate}
  \item Syntax
	\begin{enumerate}
	 \item allow {\it filename} $\mid$ {\it label} [r],[w],[x],[s];
	\end{enumerate}
  \item Meaning\\
	\begin{enumerate}
	 \item  Allow operation to file specified by permissions. 
	\end{enumerate}
  \item Specifying filename\\
	\begin{itemize}
	 \item Grab\\
	       For {\it filename} {\it directory}/* and {\it directory}** can
	be used.  For example, /var/* means, all files in /var
	directory, and /var directory. /var/** means, all files under
	/var, /var directory and files/dirs in its
	child,child's child... 	directories.
	 \item Home directories\\
	       File name that starts with  \textasciitilde ~ represents
	       home directory(Not including /root).
	       For example, ~/public\_html means,
	       /home/username/public\_html.\\
	       But in role configuration, it is different, it means home directories for users that can use role.

	\end{itemize}

  \item Meaning of permissions.
  \begin{itemize}
   \item r(Read)\\
	 Allows to read file
   \item w(Write)\\
	 Allows to write,create,delete file. Note that creation of
	 device is not allowed unless {\it allowpriv devcreate} is described.
   \item x(eXecute)\\
	 Allows to execute file.
   \item s(Search)\\ 
	 Search file tree. i.e. Get contents of directory. For file,means nothing.
  \end{itemize}
  \item Example\\
	\{\\
	domain httpd\_t;\\
	...\\
	allow /var/www/** r,s;\\
	....\\
	httpd\_t is allowed to read all files and directories under
	/var/www and its childrens.
  \item Detailed configuration support\\
	In addition to a s,r,x,w permissions, new permissions o,t,a,c,e
	can be used. Permission {\it w} is devided into those
	permissions.\\
	\begin{itemize}
	 \item o: Overwrite\\
	       Allows only writing file, not allow create,delete.
	 \item t: seTattr\\
	       Allow modify attribute of file.
	 \item a: Append\\
	       Allow append to file.
	 \item c: Create\\
	       Allow to create file.
	 \item e: Erase\\
	       Allow to delete file  
	\end{itemize}
  \item Domain execute permission\\
	{\it dx} permission means Domain Execute. If domain is defined
	for the program, program is executed in new domain.
	\begin{verbatim}
Example:
        {
          program /usr/sbin/httpd;
          allow /var/www/cgi-bin/test.cgi r,s,dx;
        }
        {
          program /var/www/cgi-bin/test.cgi;
          allow ............
        }
	\end{verbatim}
	In this case, /usr/sbin/httpd have dx permission to
	test.cgi. Domain is defined below. So, test.cgi runs as
	different domain.           
  \item Limitation about homedirectories\\
	Deny statement for individual home directory does not work.
	For example,
	\begin{verbatim}
deny /home/ynakam/public_html;
	\end{verbatim}
does not work.

 \end{enumerate}

\subsection{deny}
\begin{enumerate}
 \item Syntax\\
 deny {\it filename};
 \item Meaning\\
This is used to describe constraints for allow and, also used to cancel allow.
 \item Example
\begin{enumerate}
 \item Example 1: Describe constraints\\
   \begin{verbatim}
*In file constraints
deny /etc/shadow;

*In httpd_t.a
{
domain httpd_t;
include constraints;
allow /etc/* r,s; 
}
   \end{verbatim}
By {\it include constraints;} configuration in file constrains is
       included .
So, the above configuration is the same as following.
       \begin{verbatim}
{
domain httpd_t;
include constraints;
deny   /etc/shadow;
allow /etc/* r,s;
}
       \end{verbatim}
This means, httpd\_t have r,s permission to files in /etc. But can not
       access /etc/shadow.
To allow access to /etc/shadow, 
{\it allow /etc/shadow r,s;} should be described explicitly.
Deny is useful to prevent misconfiguration.

 \item Example 2: Cancel allow\\
\begin{verbatim}
{
domain httpd_t;
allow /etc/* r,s;
deny /etc;
\end{verbatim}
{\it allow /etc/* r,s;} is cancelled by deny /etc;

\end{enumerate}
\end{enumerate}

\subsection{Priority of allow, deny when conflict happens}
 \begin{enumerate}
  \item OR operation(When allow confilicts) \\
	When allow rule conflicts, 
	OR operation is applied.\\
	\begin{itemize}
	 \item Example
	       \begin{verbatim}
domain foo_t;
allow /var/** r;
allow /var/** s;
	       \end{verbatim}
foo\_t have r,s permission to under /var.
	       \begin{verbatim}
domain foo_t;
allow /var/run/* r;
allow /var/run/** w;
	       \end{verbatim}
foo\_t have r permission to in /var.
But for subdirectory(/var/run/xxx etc), it has w permission.

  \item Conflict between child and parent
	       \begin{verbatim}
domain foo_t;
allow /var/** r;
allow /var/run/** w;
	       \end{verbatim}

foo\_t have r permission to under /var(including subdir).
For /var/run , it has only w permission.
	\end{itemize}

  \item Cancel previous configulation(When allow/deny conflicts)\\
	When allow and deny conflicts, 
configuration that appears later survives.
       \begin{itemize}
	  \item Example
		\begin{verbatim}
domain foo_t;
allow /foo/* r,s;
deny  /foo/* ;
		\end{verbatim}
		{\it allow /foo/* r,s} is cancelled.
		\begin{verbatim}
domain foo_t;
deny  /foo/* ;
allow /foo/* r,s;
		\end{verbatim}
		{\it deny /foo/* } is cancelled.

		\begin{verbatim}
domain foo_t;
allow  /foo/bar/** r,s;
deny   /foo/** ;
		\end{verbatim}
		{\it allow /foo/bar/** r,s} is cancelled.

	\item Exception\\
		However,		
		\begin{verbatim}
domain foo_t;
deny  /foo/bar/**;
allow /foo/** r,s;
		\end{verbatim}
		{\it deny /foo/bar/**}  is {\it not} cancelled. To
		cancel deny, you have to describe allow for denied
		directory(in this case, {\it allow /foo/bar some\_permission;})
       \end{itemize}
 \end{enumerate}

\subsection{Special files}
Access to following files are special.
\begin{enumerate}
 \item  /dev/tty* /dev/pts /dev/ptmx, /dev/vcs*,/dev/vcsa*\\
	If you write allow for those file, this does nothing.
	Access control to these files must be done by allowdev. 
 \item  /proc, /sysfs, /selinux, /dev/tmpfs\\
	Allow to these files do nothing, because these files are mounted
	on filesystems that do not support xattr. See allowfs. For
	/selinux see allowpriv getsecurity.
\end{enumerate}


\subsection{Notice about links}
\subsubsection{Treatment of symbolic links}
xxx
\subsubsection{Treatment of hard links}
xxx

\section{Configuring access control to devices:allowdev}
\subsection{allowdev(1)}
 Device files must be handled carefully. Because device files are
interface to kernel. When device file is linked to driver that handles
critical information, read/write to such device will lead to leak of
confidentical informaion or break of system. Following allowdev
statements restricts access to device files.
\begin{enumerate}
 \item syntax
      \begin{enumerate}
       \item allowdev -root {\it directory};
      \end{enumerate}
 \item meaning\\
	By default, when {\it allow} statement is described to file,
	access to device files are not allowed. The directory that
	contains devices must be described in advance, by allowdev
	-root.

 \item Example\\
       \begin{verbatim}
	{ 
	domain httpd_t;
	allow /dev/* r,w;
       \end{verbatim}
       In above, httpd\_t can acesss normal files under /dev, but can
       not access device files.
       \begin{verbatim}
	{ 
	domain httpd_t;
	allowdev -root /dev;
	allow /dev/* r,w;
       \end{verbatim}
       In above, httpd\_t can access both normal files and devices under
       /dev.
       However, in permission {\it w}, creation and remove devices are not granted unless {\it allowpriv devcreate } is described. 
\end{enumerate}

\subsection{allowdev(2)}
tty devices are device files /dev/tty*, pts devices are devices under
/dev/pts. tty devices represents local login terminal, and pts devices
represents terminal in X and ssh terminal. These devices are linked to
terminal when user logs in, or open X/ssh terminal. If you can write
other users terminal device files, you can write message to his
terminal. 
In SELinux environment, tty/pts device files are given label according to
login user's role. 
So tty/pts device files should be treated differently in SPDL.

\begin{enumerate}
 \item syntax
      \begin{enumerate}
       \item allowdev -pts|-tty|-allterm open;
       \item allowdev -pts|-tty|-allterm {\it role} [r],[w];  
       \item allowdev -pts|-tty|-allterm {\it role} admin;
      \end{enumerate}
 \item meaning\\
       -tty means, tty devices. -pts means, pts devices. -allterms means
       both tty and pts devices.
       \begin{enumerate}
	\item This is usually used in role section. Allow role to have
	      its own tty/pts device. At the time of login, by login
	      program, role's tty device file is given type {\it {\it role prefix}}\_tty\_device\_t. 
	\item  Allow to read/write {\it {\it role}}'s tty
	      device. 
	\item Allow to change label of tty device, and rename, unlink.
       \end{enumerate}
       \item Special role\\
       \begin{itemize}
	\item general\\
	      this means tty/pts before labeled(The type is devtty\_t
	      and tty\_device\_t, devpts\_t, ptmx\_t). Usually, access
	      to these are harmless except {\it admin } permission.
	\item all\\
	      All other roles tty/pts
	\item vcs\\
	      This can be used only in allowdev -tty. Means vcs file(/dev/vcs*, /dev/vcsa*), these provide access to screenshot of tty terminal. 
       \end{itemize}

\end{enumerate}

\section{Configuring access control to files on misc file systems:allowfs}
 SELinux can do fine-grained access control to files on filesystems that
 suport extended-attributes, such as ext3, ext2 and xfs. For such files,
 you configure access control using {\it allow} statement. In other
 filesystems, you should configure {\it allowfs} described in this
 section.

\begin{itemize}
 \item Syntax
       \begin{enumerate}
	\item  allowfs {\it name\_of\_filesystem} [s],[r],[x],[w];\\
	       For {\it name\_of\_filesystem} {\it tmpfs sysfs autofs usbfs cdfs romfs
	       ramfs dosfs smbfs nfs proc proc\_kmsg proc\_kcore xattrfs} can be
	       used.
       \end{enumerate}      
 \item Meaning\\
       \begin{enumerate}

	\item Allow access to files in specified system. For example, {\it
	      allowfs proc s,r;} means to grant s,r access to files on proc
	      filesystem(/proc). When you see logs whose types are {\it
	      filesystem\_t }, you may have to use allowfs. This means, if you
	      find log about {\it read access to sysfs\_t is denied}, you may
	      add {\it allowfs sysfs s,r;}.
		
       \end{enumerate}
       

 \item Notice about {\it name\_of\_filesystem}
       \begin{itemize}
	\item proc filesystem\\
	      Access control to proc file system is a little
	      fine-grained. proc\_kmsg means, /proc/kmsg, proc\_kcore
	      means /proc/kcore. proc\_pid\_self means process information of
	      own process in /proc/pid/. proc\_pid\_other means proccess information for all
	      others. proc means other files on /proc. 
	\item xattrfs\\
	      This means filesystem that supports extended-attribute,
	      but not configured to use SELinux's label. For example, if
	      you format USB memory as ext3 on non-SELinux machine. Next
	      you mount the USB memory in SELinux machine, 
	      the files on it are recognized
	      as xattrfs. You have to use {\it allowfs xattrfs
	      permissions} in such case.
	\item cdfs\\
	      This corresponds to iso9660 and udf filesystem.
	\item dosfs\\
	      This corresponds to fat, vfat, ntfs.
	\item smbfs\\
	      This corresponds to cifs and smbfs.
       \end{itemize}
\end{itemize}

\section{Configuring access control to temporaly file:allowtmp}
\subsection{Why allowtmp is necessary?}
{\it allowtmp} is prepared to configure access control to temporaly
files.
Before going detail, let's see why such configuration element is necessary.
SELinux identifies files based on inode, not file name. File name based
configuration does not work correctly when inode number changes or inode
does not exist at the time of configuration(typically such files are
temporaly files).
Such files exist under  /var/run, /tmp, /var/tmp.
For example, assume following configuration exists.\\
\begin{verbatim}
domain httpd_t
allow /var/run r,s;
allow /var/run/httpd.pid r,w,s;
\end{verbatim}
At first, httpd\_t have r,w,s permission to /var/run/httpd.pid.
However, when httpd is restarted /var/run/httpd.pid is removed and
created again. In this process, inode number is changed. When inode
number changes, it inherits parent directory's permission. i.e:
 httpd\_t have r,s permission to /var/run/httpd.pid(the permission of
 /var/run). So to grant r,w,s permission to /var/run/httpd.pid, r,w,s
 permission should be given to parent directory(/var/run).
However, in this configuration, httpd\_t can r,w,s other daemons pid
files under /var/run. 
\\
In second example, when program creates files randomly under /tmp it is
a problem. Assume program  A(domain is a\_t) and program B(domain is
b\_t) creates files whose names are random under /tmp. In such
case,following configuration will be described.
\begin{verbatim}
{
domain a_t;
allow /tmp r,w;
}
{
domain b_t;
allow /tmp r,w;
}
\end{verbatim}
This means, program A can access program B's temporaly files, and
program B can access program A's temporaly files. \\
In above example, access control configuration can not be described for
individual files, but for directory what such files belongs.
If you think it is enough, folloing will not necessary :-).\\

\subsection{What is allowtmp?}
To resolve this problem, SELinux has a feature called file type
transition. {\it allowtmp } is a feature to configure file type
transition.
In file type transition, when domain creates files under some directory,
created file is given a label. The label can be named by policy.
Following is example  usage of {\it allowtmp}.
\begin{verbatim}
domain httpd_t;
allow /var/run r,s;
allowtmp -dir /var/run -name httpd_var_run_t; -(a)
allow httpd_var_run_t r,w,s; -(b)
\end{verbatim}
In (a), when httpd\_t create file under /var/run, it is labeled as {\it
httpd\_var\_run\_t}. And in (b), httpd\_t can r,w,s access to the
created file. To identify file using label name(httpd\_var\_run\_t).

\subsection{Syntax and meaning}
\begin{enumerate}
 \item Syntax
       \begin{enumerate}
	\item allowtmp -dir {\it directory} -name {\it label} {\it
	      permission};
	\item  allowtmp -fs {\it file system name} -name {\it label} {\it
	      permission};
	      {\it permission} is the same as file permission and can be omitted.
       \end{enumerate}
 \item Meaning\\
       \begin{enumerate}
	\item When domain create file under {\it directory} it is
	      labeled as {\it label} and have permission to the created
	      file specified by {\it permission}. {\it permission} can
	      be omitted. When omitted, permission can be given by {\it allow}.
	\item This is used to configure allowtmp under files that do not
	      support extended attribute, currently, this can be used
	      only for tmpfs.
	\item About label
	      \begin{itemize}
	       \item When {\it label} is {\it auto }, label is named
		     automatically based on domain and directory. For example,
		     domain is hoge\_t, and {\it directory} is /var/, label
		     name is hoge\_var\_t.
	       \item When {\it label} is {\it all} or *, it means all
		     files under {\it directory} created using allowtmp.
	      \end{itemize}           
       \end{enumerate}       
 \item Example\\
       \begin{verbatim}
domain httpd_t	;
allowtmp -dir /var/run -name auto r,w,s;
       \end{verbatim}
Files created under /var/run by httpd\_t is labeled as
       httpd\_var\_run\_t and httpd\_t can r,w,s access to such files.

\begin{verbatim}
domain httpd_t
allowtmp -dir /var/run -name auto r,w,s;
domain named_t
allowtmp -dir /var/run -name auto r,w,s;
domain initrc_t;
allowtmp -dir /var/run -name all r,w,s;
\end{verbatim}
Files created under /var/run by httpd\_t is labeled as
       httpd\_var\_run\_t and httpd\_t can r,w,s access to such
       files(named\_t can not access).
Files created under /var/run by named\_t is labeled as
       named\_var\_run\_t and named\_t can r,w,s access to such
       files(httpd\_t can not access)
initrc\_t can r,w,s access to above files because {\it -name all} is
       specified. {\it -name all} is used to administate files created
       by allowtmp.

\end{enumerate}



\section{Configuring access control to network:allownet}
{\it allownet} statements is prepared to configure network access control.
It can configure access control to  port , netif(Network
Interface), node(IP address) and inheritance of socket.

\subsection{Port usage}
\begin{enumerate}
 \item Syntax\\
 allownet -protocol {\it protocol} -port {\it port number} {\it permission};\\
 {\it protocol}: tcp,udp can be specified, splitted by camma.\\
 {\it port number}: number and {\it -1023} and {\it 1024-} , and * can
       be described, splitted by camma.\\
 {\it permission}: {\it client} or {\it server} splitted by camma can be
       described
 \item Meaning\\
       Allow permissions to be TCP/UDP server/client using port. Port number {\it
       -1023} means, all unused ports under 1023. {\it 1024-} means all
       unused ports after 1024. * means all ports. 
 \item Note about udp server\\
       If you describe allownet -protocol udp -port xxx server;
       The domain also behave as client to port number over 1024.
 \item Example
\begin{verbatim}
domain httpd_t;
# httpd_t can be server using port 80 and 443.
allownet -protocol tcp -port 80,443 server;
# httpd_t can use TCP/UDP 3306 service(MySQL) as client.
allownet -protocol tcp,udp 3306 client;
#Socket usage must be allowd to use port
allownet -protocol tcp,udp use;
\end{verbatim}
\end{enumerate}

\subsection{Usage of RAW socket}
Usage of RAW socket must be restricted, because RAW socket can be used
for IP spoofing and easedropping.
\begin{enumerate}
 \item Syntax\\
       allownet -protocol raw use;\\
        {\it permission}: {\it client} or {\it server} or * splitted by camma can be
       described.
 \item Meaning\\
       The domain is allowed to use RAW socket.
\end{enumerate}

\subsection{Usage of Network Interface(netif) and IP address(node)}
Usage of netif/node is allowed by this. In default policy, it is allowed
to all domains.
\begin{enumerate}
 \item Syntax\\
       \begin{enumerate}
	\item allownet -protocol {\it protocol} -netif {\it name of NIC} {\it
	      permission};\\
	      {\it protocol}: tcp,udp,raw and * can be specified, splitted by camma.\\
	      {\it name of NIC}: NIC name(such as lo,eth0,eth1) splitted by camma.\\
	      {\it permission}: {\it send} or {\it recv} splitted by camma can
	      be described.       
	\item allownet -protocol {\it protocol} -node {\it address} {\it
	      permission};\\
	      {\it protocol}: tcp,udp,raw and * can be specified, splitted by camma.\\
	      {\it address}: {\it ipv4address}/{\it netmask} or * splitted by
	      camma. Example: 192.168.0.1/255.255.255.0 . And * means all address.\\
	      {\it permission}: {\it send} or {\it recv} splitted by camma can
	      be described. 
       \end{enumerate}
 \item Meaning\\
       \begin{enumerate}
	\item    Allows to send or receive packet to/from NIC. 
	\item  Allows to send or receive packet to/from IP address. 
       \end{enumerate}
 \item Example\\

\begin{verbatim}
{
domain httpd_t;
allownet -protocol tcp  use;
allownet -protocol tcp -port 80 server;
allownet -netif eth0 send,recv;
}
--> httpd_t can use tcp socket and be server using TCP 80 port.
And can send/recv packet to/from eth0.
\end{verbatim}
\end{enumerate}

\subsection{Inheriting socket from other domain}\label{sec:socket}
Following syntax allows using socket of other domain. It is rare to configure.
\begin{enumerate}
 \item  Syntax 
	\begin{enumerate}
	 \item allownet -protocol {\it protocol}  -domain {\it domain} use;
	       {\it protocol}, tcp,udp can be specified, splitted by camma.
	\end{enumerate}

 \item Meaning\\
	\begin{enumerate}
	 \item This enables to restrict inheriting socket from other
	       domain. This configures from where the domain can
	       inherit socket. When {\it domain} is {\it self}, the domain can
	       use socket which is created by its own domain. 
	\end{enumerate}
\item Example
\begin{verbatim}
domain foo_t;
# foo_t can inherit UDP socket from bar_t
allownet -protocol udp -domain bar_t; 
\end{verbatim}
\end{enumerate}
\section{Configuring access control to process communication:allowcom}
\subsection{allowcom (IPC)}
\begin{enumerate}
 \item Syntax(1)\\
       allowcom -ipc {\it to domain} [r],[w];
 \item Syntax(2)\\
       allowcom -unix$\mid$-sem$\mid$-msg$\mid$-msgq$\mid$-shm$\mid$-pipe {\it to
       domain} [r],[w];
 \item Meaning\\
       Allow to communicate with {\it  to domain } by specified IPC.\\
       If {\it to domain } is {\it self}, this means IPC within
       domain. If {\it  to domain } is {\it *} the domain can IPC
       to every domain.
       -ipc is allowing all kinds of IPCs, it is simplified
       configuration. If you want to specify specific kind of ipc, you
       can use following.      
       -unix is unix domain socket, -sem is semaphore, -msg is message,
       -msgq is message queue, -shm is shared memory, -pipe is pipe.
\end{enumerate}

\subsection{allowcom(Signal)}
\begin{enumerate}
 \item Syntax\\
       allowcom -sig {\it to domain} [c],[k],[s],[n],[o];
 \item Meaning\\
       Allow to send signal to {\it to domain}. [c] is sigchld, [k] is
       sigkill, [s] is sigstop, [n] is signull, [o] is other signals. signull is not
       supported, this means all domains are allowed to use signull.
\end{enumerate}



\section{Configuring access control other administrative access
  rights:allowpriv}
\begin{itemize}
 \item Syntax \\
       allowpriv {\it string};\\
       {\it string} is name  of privilage. It is described in next section.
 \item Meaning \\
       Allow access rights represented by        {\it string}.
\end{itemize}
Next, 
what can be configured for {\it string} can be categorized into
following.
\begin{itemize}
 \item POSIX capability\\
 \item Related to kernel\\
 \item Related to SELinux operations\\
 \item Others
\end{itemize}


\subsection{allowpriv: POSIX capability}
Strings that begin with {\it cap\_} is posix capability. You can see
detailed meaning by man capabilities.

\subsubsection{Capabilities that can not be configured}
Following POSIX capabilities can not be configured, because it can be
configured in different place. \\
\begin{itemize}
 \item CAP\_NET\_BIND\_SERVICE\\
       This restricts usage of wellknown ports, but by allownet, you can
       configure better. So this is omitted.
 \item CAP\_MKNOD\\
       This is allowed in allowpriv devcreate.
 \item CAP\_AUDIT\_WRITE\\
       Operations that is restricted by this is the same as
       allowpriv audit\_write ,so this  is omitted.
 \item CAP\_AUDIT\_CONTROL\\
       Operations that is  restricted by this is the same as allowpriv
       audit\_control, so this is  omitted.
\end{itemize}

\subsubsection{Configurable capabilities}
\begin{itemize}
 \item cap\_sys\_pacct\\
       Configures kernel accounting(see acct(2)).
 \item cap\_sys\_module\\
       Allows to install kernel module.
 \item cap\_net\_admin\\
       Allow capability {\it CAP\_NET\_ADMIN}(Such as 
       administrate NIC, route table). 
 \item cap\_sys\_boot\\
       Allow capability{\it CAP\_SYS\_BOOT}. This means allow the
       usage of reboot system call.
 \item cap\_sys\_rawio\\        
       Allow capability {\it CAP\_SYS\_RAWIO}.This means usage of
       ioperm, iopl system call and access to /dev/mem.
 \item cap\_sys\_chroot\\
       Allow to use chroot.
 \item cap\_sys\_nice\\
       Allow capability {\it CAP\_SYS\_NICE}. This means process scheduling.
 \item cap\_sys\_resource\\
       Allow capability {\it CAP\_SYS\_RESOURCE}. This means usage
       of rlimit etc.
 \item cap\_sys\_time\\
       Allow capability {\it CAP\_SYS\_TIME}. Thie means modify
       system clock.
 \item cap\_sys\_admin\\
       The same as POSIX capability {\it CAP\_SYS\_ADMIN}. This
       permissions overlaps other permissions, so if you allow
       this, not so serious problem. By denying this,
       it can restrict sethostname and some ioctl operations.
 \item cap\_sys\_tty\_config\\
       The same as capability {\it CAP\_TTY\_CONFIG}. Change
       keyboard configuration, and usage of vhangup call. 
 \item cap\_ipc\_lock\\ 
       Allow capability {\it CAP\_IPC\_LOCK}. This means to lock
       memory.
 \item cap\_dac\_override\\ 
 \item cap\_dac\_read\_search \\
 \item cap\_setuid 
 \item cap\_setgid \\
 \item cap\_chown \\
 \item cap\_setpcap\\
 \item  cap\_fowner\\
 \item  cap\_fsetid \\
 \item cap\_linux\_immutable\\
 \item cap\_sys\_ptrace\\
\item cap\_lease\\
\item cap\_ipc\_owner\\
\item cap\_kill\\

\end{itemize}


\subsection{allowpriv: related to kernel}
Configures priviledges to communicate and administrate kernel. 
Following strings can be used.
       \begin{enumerate}
	\item netlink\\
	      Allows to communicate with kernel by netlink socket. 
	\item klog\_read\\
	      Allows to read kernel messages by syslog(2) call. Usually
	      it is required to use dmesg command.
	\item klog\_adm\\
	      Allows to change configuration of kernel message output.
	\item audit\_read\\
	      Allows to read status and configuration of kernel audit
           subsystem.
	\item audit\_write\\
	      Allows to send log message to audit subsystem in
	      kernel.
	\item audit\_adm\\
	      Change configuration of kenel audit subsystem.
	\item klog\_adm\\
	      Allows to change configuration of audit in kernel. The
	      same as capability audit\_control,sys\_pacct.
	      
       
       \end{enumerate}

\subsection{allowpriv: related to SELinux operations}
       Allow priviledges to administrate SELinux.
       \begin{enumerate}
	\item relabel\\
	      Allow to relabel all files. You must also allow
	      getsecurity and allowpriv search.
	\item part\_relabel\\
	      Allow to relabel files that the domain can write. You must
	      also allow getsecurity. 
	\item setfscreate\\
	      This is necessary only applications that use SELinux API(setfscreatecon).
	\item getsecurity\\
	      Allow to get security policy decisions, by accessing /selinux.
	\item setenforce\\        
	      Allow to toggle enforcing/permissive mode.
	\item load\_policy\\                    
	      Allow to load policy to kernel.
	\item setsecparam\\
	      Change performance parameter of SELinux via /selinux/avc
	\item getsecattr\\
	      Get security information(such as domain, stored in /proc/pid/attr) of other processes.
       \end{enumerate}

\subsection{allowpriv: other privileges}
       Allow other priviledges.
       \begin{enumerate}

	\item quotaon\\         
	      Allow to quotaon.
	\item mount\\         
	      Allow to mount device.

	\item unlabel\\
	      Allow full access to unlabeled files(Files labeled as
	      unlabeled\_t).

	\item devcreate\\
	      Allow to create device files in directory that the domain can write.
	      Without this, a process can not create device
	      file on a directory even it is configured writable.
	\item setattr\\
	      Allow to setattr to files that the domain can s
	      access. Without this setattr permission is granted in w permission.	\item search\\
	      Allow s permission to all files.
	\item read\\
	      Allow r permission to all files.
	\item write\\
	      Allow w permission to all files.
	\item all\\
       \end{enumerate}     

\subsection{denypriv}
This can be used to cancel allowpriv configuration.


\end{document}
