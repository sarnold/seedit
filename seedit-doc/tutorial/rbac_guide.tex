\documentclass{article}
\title{seedit RBAC(Role Based Access Control) guide (for Ver 2.0))}
\author{Yuichi Nakamura \thanks{ynakam@gwu.edu}}
\begin{document}
\def\labelenumi{(\theenumi)}
\maketitle
\tableofcontents
\newpage


\section{What is RBAC}
SPDL supports configuration of RBAC(Role Based Access Control).
In default policy, the domain for login user is unconfined\_t.
So, the behavior of user is not confined by SELinux.\\

To increase security of login user, RBAC is useful.

xxxxx

\section{Enable RBAC}
To enable RBAC, use following command.
\begin{verbatim}
# seedit-rbac on
(It takes some minutes)
# reboot
\end{verbatim}
By above commands, files necessary to configure RBAC is moved from
/etc/seedit/policy/extra to /etc/seedit/policy, and seedit-load is
run.Reboot is necessary because some domains become invalid, to fix this
you have to reboot.\\

To disable RBAC, use following.
\begin{verbatim}
# seedit-rbac off
# reboot
\end{verbatim}

\section{Default RBAC Configuration}
Following 3 roles are defined by default.

\begin{itemize}
 \item sysadm\_r\\
       It is role for administrator. It can work as unconfined
       domain sysadm\_t. By default, only root can login as the role.
 \item staff\_r \\
       It is role, to do not administrative work for administrative
       user.
       By default, only root can login as the role.
 \item user\_r \\
       It is a role for normal users. By default, user\_u can login as
       root.
       user\_u is a user that is not configured RBAC, by default users
       except root.
       
\end{itemize}

Attention!!\\
{\it su command can not be used, except sysadm\_r}.


\section{Login by RBAC}
\subsection{Check role}
When you enable RBAC, role are given to login user. 
You can see it by id command.\\
When you login as root user,  staff\_r role is given.\\
Let's see it by id command.
\begin{verbatim}
# id 
uid=0(root) gid=0(root) groups=0(root),1(bin),2(daemon),
3(sys),4(adm),6(disk),10(wheel) 
context=root:staff_r:staff_t
\end{verbatim}
\begin{verbatim}
context=root:staff_r:staff_t	
\end{verbatim}
shows role, staff\_r is role. User shell is given domain according to
role, in this case staff\_t. Inside SELinux, domain is used for access
control.\\
staff\_r is given little access rights. You can not do any
administration work in this role.\\
For example, you can not access homepage.
\begin{verbatim}
# cat /var/www/html/index.html
Permission denied
\end{verbatim}

\subsection{Change role}
To do administrative work, you have to switch role to sysadm\_r role.
You can do it by newrole command, like below.
\begin{verbatim}
# newrole -r sysadm_r
Authenticating root
Password:
\end{verbatim}
You have to enter password of current user(this case root).
Then check role by id command.
\begin{verbatim}
# id 
uid=0(root) gid=0(root) groups=0(root),1(bin),2(daemon),
3(sys),4(adm),6(disk),10(wheel) 
context=root:sysadm_r:sysadm_t
\end{verbatim}
Role is  sysadm\_r. Domain of user shell is sysadm\_t. sysadm\_t is
unconfined domain, so you can do any work.\\
To switch role, the user must be allowed to use the role, if the user is
not allowed to use the role, newrole will fail.\\
To allow user to use role, see next section.

\section{Configuration elements for RBAC}
\subsection{role and user statements}
To configure RBAC, there are SPDL configuration elements, {\it role} and
{\it user}.
\begin{itemize}
 \item role {\it name  of role}\\
       This means configurations is going to be done for specified
       role. Access right is given to corresponding domain. And the
       configuration filename must be {\it role name}.sp.
 \item user {\it user name}\\
       This allows user to use role.  user\_u user name is special, it is
       the default role of users that is not configured to use domain.
\end{itemize}

\subsubsection{Example}
Let's see example.
\begin{verbatim}
1:{
2:role webmaster_r;
3:user webmaster;
4:allow /var/www/** r,w,s;
}	
\end{verbatim}

Line 2 means, configurations between \{\} is for webmaster\_r role.\\
Line 3 means, user name webmaster can use webmaster\_r role.\\
Following, access rights are given to domain webmaster\_t
domain(Remember that webmaster\_r role behaves as webmaster\_t domain in SELinux system).
Line 4 means, webmaster\_t domain(This equals user that logined as
webmaster\_r role)is allowed to read, write under /var/www.\\
\subsubsection{user\_u user name}
Let's see example  of user\_u user name.\\

\begin{verbatim}
Assume all configurations for RBAC are following.
* In sysadm_r.sp
{
role sysadm_r;
user root;
..
}	
* In webmaster_r.sp
{
role webmaster_r;
user webmaster
..
}
* In user_r.sp
{
role user_r;
user user_u;
..
}
\end{verbatim}
In above, 3 roles are configured. 
You can see, user root and webmaster are assigned role.
In this case user\_u is {\it all users except root and webmaster}.


\subsection{Home directories}

To configure access control to home directories, we could use
\textasciitilde / . In normal domain, it means all users home
directoies. But, for configuration of role, the meaning is different.\\





\section{Creating new role}
The best way to understand RBAC is to create  new role. Let's see it by
example.
Here, we will create new role for webmaster, the name is webmaster\_r.
And assign the role to user whose name is webmaster.
\subsection{Create uid=0 user}
The uid for  user that does some administration work, must be 0.
It is to pass Linux permission check.\\
Following commands create user webmaster as uid=0.
\begin{verbatim}
# useradd -u 0 -o webmaster
# passwd webmaster
\end{verbatim}

\subsection{Create template}
You can create template configuration by seedit-template command.\\
The usage is following.
\begin{verbatim}
seedit-template -r <role> -u <user> -o <output directory>
\end{verbatim}
If you specify -o option, configuration is written to file, before
writing to file, run command without -o option to make sure.\\


Following is example of generating configration for webmaster\_r role.

\begin{verbatim}
# seedit-template -r webmaster_r -u webmaster
{
role webmaster_r;
user webmaster;
include user_common.sp;
include common-relaxed.sp;
allow ~/** r,w,s;
}
\end{verbatim}
Template configuration is generated. Before writing to file, let's see
the meaning.




%%restorecon ��Ȥ���褦�ˤ���Ȥ� allowpriv part_relabel


\end{document}