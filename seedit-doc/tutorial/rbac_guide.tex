\documentclass{article}
\title{SELinux Policy Editor(seedit) RBAC guide (for Ver 2.0))}
\author{Yuichi Nakamura \thanks{ynakam@gwu.edu}}
\begin{document}
\def\labelenumi{(\theenumi)}
\maketitle
\tableofcontents
\newpage


\section{What is RBAC}
SPDL supports configuration of RBAC(Role Based Access Control).
In default policy, the domain for login user is unconfined\_t.
So, the behavior of user is not confined by SELinux.\\

To increase security of login user, RBAC is useful.

xxxxx

\section{Enable RBAC}
To enable RBAC, use following command.
\begin{verbatim}
# seedit-rbac on
(It takes some minutes)
# reboot
\end{verbatim}
By above commands, files necessary to configure RBAC is moved from
/etc/seedit/policy/extra to /etc/seedit/policy, and seedit-load is
run.Reboot is necessary because some domains become invalid, to fix this
you have to reboot.\\

To disable RBAC, use following.
\begin{verbatim}
# seedit-rbac off
# reboot
\end{verbatim}

\section{Default RBAC Configuration}
Following 3 roles are defined by default.

\begin{itemize}
 \item sysadm\_r\\
       It is role for administrator. It can login as unconfined
       domain sysadm\_t. By default, only root can login as the role.
 \item staff\_r \\
       It is role, to do not administrative work for administrative
       user.
       By default, only root can login as the role.
 \item user\_r \\
       It is a role for normal users.
       
\end{itemize}

su command can not be used, except sysadm\_r.

\section{Login by RBAC}



\section{Creating new role}

\end{document}